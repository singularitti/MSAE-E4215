% !TEX program = pdflatex
% !TEX encoding = UTF-8
% !TEX spellcheck = en_US
\documentclass[12pt]{article}

\usepackage[margin=1in]{geometry}
\usepackage{amsmath,amsthm,amssymb}
\usepackage{graphicx}
\usepackage{hyperref} % Uso de links
\usepackage[version=4]{mhchem}
\usepackage{siunitx}
\usepackage{enumerate}
\usepackage{titlesec}
\usepackage{booktabs}
\usepackage{bm}

\titleformat{\subsection}
  {\normalfont\large\bfseries}{}{0em}{}

\setlength{\parindent}{0em}

\begin{document}

% --------------------------------------------------------------
%                         Start here
% --------------------------------------------------------------

\title{Mechanical Properties of Materials (MSAE 4215), Spring 2019\\ Homework 2 Solutions}
\author{Qi Zhang}
\date{\today}

\maketitle

\tableofcontents
\listoffigures
\listoftables

\section{Problems}
\subsection{2.1}
Consider a solid cylinder, radius $R$, length $l$, density $\rho$ standing up on its end,
so that its total height (in $x_3$) above the pavement is $l$. Assume normal stresses only.
Derive an expression for the stress in the material as a function of $x_3$, and
plot $\sigma_{33}(x_3)$. How tall can a solid column of concrete ($\rho = \SI{2}{\gram\per\cubic\centi\meter}$)
be made before the maximum shear stress exceeds the critical shear stress for
fracture, $\tau_c = \SI{10}{\mega\pascal}$? Take $\sigma_{33} \sim \tau_\text{max}$ at fracture.
(Recall that the total forces exerted on the body have to sum to zero.)

\textbf{Solution:}





\subsection{2.2}
Consider a small sphere of \ce{NdFeB} permanent magnet material in a large,
uniform, superconducting magnetic field of \SI{10}{\tesla}. If its magnetization $M$
is initially orthogonal to the applied field $\mu_0 H$ when the field is (instantaneously)
turned on, does the sphere shatter? Take $\tau_c = \SI{50}{\mega\pascal}$? and
$\mu_0 M_s = \SI{0.5}{\tesla}$.

\textbf{Solution:}


\subsection{2.3}
There is a longstanding proposal for a space elevator, as pictured in
Figure $2.5$. If a mass $M$ can be attached to a long cable, length $l$, and $l \gg R$,
where $R$ is the radius of the earth, gravity is less strong than the centrifugal force,
the cable might be supported.
1) Assume that there is no counterweight ($M = 0$). For a cable density $\rho$,
how long does the cable need to be for it to stand up (i.e. all sections in tension)?
Estimate $l/R$ for a density of $\rho = \SI{2}{\gram\per\cubic\centi\meter}$,
appropriate for carbon fiber. 2) Plot the stress in the cable, and calculate the
maximum stress.

\textbf{Solution:}



\subsection{3.1}
For a solid cylinder under uniaxial normal stress along $x_2$, $\sigma_{22} = \sigma$,
\begin{itemize}
    \item Find the shear stress $\sigma_{\alpha\beta}$ for orthonormal ($\alpha$, $\beta$, $x_3$)
          and $\alpha$, $\beta$ rotated by $\theta_3$ about $x_3$ with respect to $x_1$, $x_2$.
    \item Find the shear stress $\sigma_{\alpha\beta}$ for a general rotation $\theta$,
          and show that it is maximum for $\theta = \pi/4$.
\end{itemize}


\subsection{3.2}
Consider the following biaxial stress state: $\sigma_{11} = \SI{300}{\mega\pascal}$,
$\sigma_{22} = \SI{100}{\mega\pascal}$, $\sigma_{12} = \SI{100}{\mega\pascal}$,
with $\alpha$, $\beta$ axes defined as before
\begin{itemize}
    \item Determine the rotation $\theta_3$ such that $\alpha$, $\beta$ are principal axes.
    \item Determine the principal stresses $\sigma_{\alpha\alpha}$, $\sigma_{\beta\beta}$.
    \item Determine the $\theta_3$ for maximum in-plane shear and the magnitude of the maximum $\sigma_{\alpha\beta}$.
\end{itemize}


\subsection{3.3}
Take the following stress tensor:
\begin{equation}
    [\sigma] = \begin{bmatrix}
        0           & \sigma_{12} & 0           \\
        \sigma_{12} & \sigma_{22} & \sigma_{23} \\
        0           & \sigma_{23} & 0
    \end{bmatrix}
\end{equation}
Note that no principal stress axis is known \textit{a priori}.
\begin{itemize}
    \item Write an expression for the principal stresses. How many nonzero principal stresses are there?
    \item Verify that the dilation $\Delta$ is invariant on transforming to the principal axes.
    \item Solve the principal stresses for $\sigma_{22} = \SI{200}{\mega\pascal}$,
          $\sigma_{12} = \sigma_{23} = \sqrt{2} \cdot \SI{100}{\mega\pascal} \sim \SI{141}{\mega\pascal}$.
    \item Determine the principal axes. If there is plane stress in the transformed coordinates, determine the normal to the plane.
\end{itemize}

% --------------------------------------------------------------
%     You don't have to mess with anything below this line.
% --------------------------------------------------------------

\end{document}