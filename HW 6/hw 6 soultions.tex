% !TEX program = pdflatex
% !TEX encoding = UTF-8
% !TEX spellcheck = en_US
\documentclass[12pt]{article}

\usepackage[margin=1in]{geometry}
\usepackage[utf8]{inputenc}
\usepackage{amsmath,amsthm,amssymb}
\usepackage{graphicx}
\usepackage{hyperref} % Uso de links
\usepackage[version=4]{mhchem}
\usepackage{siunitx}
\usepackage{titlesec}
\usepackage{booktabs}
\usepackage{multicol}
\usepackage{multirow}
\usepackage{bm}
\usepackage{miller}
\usepackage{datetime}

\titleformat{\subsection}
  {\normalfont\large\bfseries}{}{0em}{}

\settimeformat{hhmmsstime}

\setlength{\parindent}{0em}

%\DeclareMathOperator{\tr}{tr}

\begin{document}

% --------------------------------------------------------------
%                         Start here
% --------------------------------------------------------------

\title{Mechanical Properties of Materials (MSAE 4215), Spring 2019\\ Homework 5 Solutions}
\author{Qi Zhang}
\date{\today, \currenttime}

\maketitle

\tableofcontents
\listoftables

\section{Problems}
Please track the time above to see whether this file is updated.

\subsection{6.3}
Using the Python function \texttt{sprime}, show that an average of $100$
random orientations for the stress/strain axis $l_i$ converges to the Reuss average.

\textit{Hint:} use the function
\texttt{2*np.pi*np.random.random((100,2))} to create random
values for spherical angles $\theta$, $\phi$ and generate $l_i$ from these angles.

\textbf{Solutions:}

The analytical result of the Reuss average is given by equation $(6.120)$:
\begin{equation}
  \bar{s}_{11} = \frac{ 3 }{ 5 } s_{11} + \frac{ 2 }{ 5 } s_{12} + \frac{ 1 }{ 5 } s_{44}
  = \SI{9.14}{\per\tera\pascal}.
\end{equation}
What we need to do is to compare this value with the computer-program-sampled-average.
The rest of this problem is solved on the IPython notebook.


\subsection{6.4}\label{ssec:64}
Derive an expression for the p-wave modulus $M$ for different directions
of uniaxial strain defined by $l_i$.

\textbf{Solutions:}

The p-wave modulus is defined as the ratio of longitudinal stress to longitudinal
strain for a uniaxial strain state. Quantitatively, we can state this as
$\sigma_{11} = M \varepsilon_{11}$, while $\varepsilon_{22} = \varepsilon_{33} = 0$.
Since we are dealing with a well defined
strain state, $\sigma_{ij} = c_{ijkl} \varepsilon_{kl}$, we can write:
\begin{equation}
  \sigma_{11} = c_{1111} \varepsilon_{11}.
\end{equation}
Therefore,
\begin{equation}
  M \equiv c_{1111} = c_{11}.
\end{equation}
Now to find the p-wave modulus for different directions of uniaxial strain we can
follow the procedure outlined in notes.
As a side-note, the final answer is given already in equation $(6.108)$.

Recall the tensor transformation for a general fourth-rank tensor:
\begin{equation}
  T_{ijkl}' = a_{im} a_{jn} a_{ko} a_{lp} T_{mnop}.
\end{equation}
Rewriting this for the p-wave modulus term, $c_{1111}$, in arbitrary direction:
\begin{equation}
  c_{1111}' = a_{1m} a_{1n} a_{1o} a_{1p} c_{mnop}.
\end{equation}
Recall from Chapter $3$, $\mathrm{ a }$ is just a rotation matrix, written generically as:
\begin{equation}
  \mathrm{ a } =
  \begin{bmatrix}
    a_{11} & a_{12} & a_{13} \\
    a_{21} & a_{22} & a_{23} \\
    a_{31} & a_{32} & a_{33}
  \end{bmatrix}.
\end{equation}
Again as a reminder this matrix $\mathrm{ a }$,
for a forward transformation, has elements $a_{ij}$ which represents the cosine of the angle
between new axis $\hat{x}_i'$ and old axis $\hat{x}_j$.
Thus,
\begin{equation}
  a_{ij} = \hat{x}_i' \cdot \hat{x}_j.
\end{equation}

Hence, $a_{1m}$ is merely the first row of the rotation matrix which relates one of the new axes,
specifically $\hat{x}_1'$, to all three of the old axes, $\hat{x}_1$, $\hat{x}_2$, and $\hat{x}_3$.
Now we define,
\begin{equation}
  l_i = a_{1i} = \hat{x}_1' \cdot \hat{x}_i, \text{ for } i = 1, 2, 3.
\end{equation}
Thus, $l_i$ is just the directional cosine between
the new $\hat{x}_1'$ axis and the old $\hat{x}_i$ for $i = 1, 2, 3$.
Why do we refer to this as a directional cosine?
Simple, the definition of the dot product between two unit-vectors, whose norm are one,
\begin{equation}
  \| \hat{x}_i \| = 1,
\end{equation}
is
\begin{equation}
  \hat{x}_i' \cdot \hat{x}_j = \| \hat{x}_i' \| \| \hat{x}_j \| \cos \theta = \cos \theta,
\end{equation}
where $\theta$ is the angle between the two vectors.

Then
\begin{equation}
  c_{1111}' = l_{m} l_{n} l_{o} l_{p} c_{mnop}.
\end{equation}

For a cubic material we only need to consider three types of terms for $c_{mnop}$.
So
\begin{equation}
  c_{1111}' = (l_1^4 + l_2^4 + l_3^4) c_{11} + 2(l_1^2 l_2^2 + l_1^2 l_3^2 + l_2^2 l_3^2) c_{12} + 4(l_1^2 l_2^2 + l_1^2 l_3^2 + l_2^2 l_3^2) c_{44}.
\end{equation}

Defining
\begin{equation}
  Q = l_1^2 l_2^2 + l_1^2 l_3^2 + l_2^2 l_3^2,
\end{equation}
follow the same steps on notes, we have
\begin{equation}
  M = c_{1111}' = c_{11}' = c_{11} - 2(c_{11} - c_{12} - 2 c_{44}) Q.
\end{equation}




\subsection{6.5}
Modify the Python function to calculate the p-wave modulus,
according to your result in the last problem.  Plot the modulus for a
full rotation of $2\pi$ in the \hkl(100) and \hkl(111) planes, and be sure to show that
your computed result agrees with analytic results for some low-index orientations.

\textbf{Solutions:}

See the associated IPython notebook, or the PDF file. Thanks to Michael Carter's contribution.
To compare the result with question \ref{tab:lbal}, compute several $Q(\theta)$ by hand.

\subsection{6.6}
Using the Landolt--Bornstein tables values, calculate the Voigt, Reuss,
and Voigt--Reuss--Hill average isotropic Young's moduls $E$ for \ce{Al}.

\textbf{Solutions:}

\begin{table}[h]
  \centering
  \caption{Landolt-Bornstein tables for \ce{Al}.}
  \begin{tabular}{cccccccc}
    \toprule
    \multirow{2}[4]{*}{Element}  & $s_{11}$                                  & $s_{44}$ & $s_{12}$                              &  & $c_{11}$ & $c_{44}$ & $c_{12}$ \\
    \cmidrule{2-4}\cmidrule{6-8} & \multicolumn{3}{c}{\si{\per\tera\pascal}} &          & \multicolumn{3}{c}{\si{\giga\pascal}}                                     \\
    \midrule
    \ce{Al}                      & 16                                        & 35.3     & -5.8                                  &  & 108      & 28.3     & 62       \\
    $s(n=10)$                    & 0.5                                       & 0.2      & 0.3                                   &  & 2        & 0.2      & 2        \\
    \bottomrule
  \end{tabular}%
  \label{tab:lbal}%
\end{table}%
The elastic constants from the Landolt-Bornstein tables for \ce{Al} are listed in table \ref{tab:lbal}.


So the Voigt-averaged Young's modulus $E_\text{V}$ is
\begin{equation}
  E_\text{V} = \frac{ \left( 0.4 (c_{11}-c_{12})+1.2 c_{44}\right)\left(c_{11}+2 c_{12}\right) }{ \left( 0.8 c_{11} + 1.2 c_{12} + 0.4 c_{44}\right) }
  \approx \SI{70.58}{\giga\pascal}.
\end{equation}

The Reuss-averaged Young's modulus $E_\text{R}$ is
\begin{equation}
  E_\text{R} = \frac{5}{3 s_{11} + 2 s_{12} + s_{44}} \approx \SI{0.06974}{\tera\pascal} = \SI{69.74}{\giga\pascal}.
\end{equation}

The Voigt--Reuss--Hill-averaged Young's modulus $E_\text{VRH}$ is
\begin{equation}
  E_\text{VRH} = \frac{ E_{V} + E_{R} }{ 2 } = \SI{70.16}{\giga\pascal}.
\end{equation}

\subsection{6.7}
Derive the Young's modulus for the Reuss average of
a \hkl[111]-fiber-textured polycrystalline cubic solid.

\textbf{Solutions:}

Take any two orthogonal unit vectors on the plane perpendicular to \hkl[1 1 1],
say, $\frac{ 1 }{ \sqrt{6} }\hkl[-1 -1 2]$ and $\frac{ 1 }{ \sqrt{2} }\hkl[-1 1 0]$.
Compose a vector $\bm{l}$ which satisfies
\begin{equation}
  \bm{l} = \frac{ 1 }{ \sqrt{6} } \hkl[-1 -1 2] \sin \theta + \frac{ 1 }{ \sqrt{2} } \hkl[-1 1 0] \cos \theta =
  \begin{bmatrix}
    -\frac{ 1 }{ \sqrt{6} } \sin \theta - \frac{ 1 }{ \sqrt{2} } \cos \theta \\
    -\frac{ 1 }{ \sqrt{6} } \sin \theta + \frac{ 1 }{ \sqrt{2} } \cos \theta \\
    \frac{ 2 }{ \sqrt{6} } \sin \theta
  \end{bmatrix}
\end{equation}
where $0 \leq \theta \leq 2 \pi$.
Since
\begin{equation}
  s_{11}' = s_{11} - 2(s_{11} - s_{12} - \frac{ s_{44} }{ 2 }) Q,
\end{equation}
where
\begin{equation}
  Q = l_1^2 l_2^2 + l_1^2 l_3^2 + l_2^2 l_3^2 \equiv \frac{ 1 }{ 4 },
\end{equation}
so
\begin{equation}
  s_{11}' \equiv s_{11} - \frac{ 1 }{ 2 } \Big(s_{11} - s_{12} - \frac{ s_{44} }{ 2 }\Big)
  = \frac{ 1 }{ 2 } \Big(s_{11} + s_{12} + \frac{ s_{44} }{ 2 }\Big).
\end{equation}
Since $s_{11}'$ is a constant, its average is just itself. Therefore,
\begin{equation}
  \bar{s}_{11}' = s_{11} = \frac{ 1 }{ 2 } \Big(s_{11} + s_{12} + \frac{ s_{44} }{ 2 }\Big).
\end{equation}
So its Young's modulus for the Reuss average is
\begin{equation}
  E = \frac{ 1 }{ \bar{s}_{11}' } = \frac{4}{2 s_{11} + 2 s_{12} + s_{44} }.
\end{equation}

% \bibliographystyle{unsrt}
% \bibliography{ref}

% --------------------------------------------------------------
%     You don't have to mess with anything below this line.
% --------------------------------------------------------------

\end{document}