% !TEX program = pdflatex
% !TEX encoding = UTF-8
% !TEX spellcheck = en_US
\documentclass[12pt]{article}

\usepackage[margin=1in]{geometry}
\usepackage[utf8]{inputenc}
\usepackage{amsmath,amsthm,amssymb}
\usepackage{graphicx}
\usepackage{hyperref} % Uso de links
\usepackage[version=4]{mhchem}
\usepackage{siunitx}
% \usepackage{enumerate}
\usepackage{titlesec}
% \usepackage{booktabs}
% \usepackage{bm}
\usepackage{miller}
\usepackage{datetime}

\titleformat{\subsection}
  {\normalfont\large\bfseries}{}{0em}{}

\settimeformat{hhmmsstime}

\setlength{\parindent}{0em}

%\DeclareMathOperator{\tr}{tr}

\begin{document}

% --------------------------------------------------------------
%                         Start here
% --------------------------------------------------------------

\title{Mechanical Properties of Materials (MSAE 4215), Spring 2019\\ Homework 5 Solutions}
\author{Qi Zhang}
\date{\today, \currenttime}

\maketitle

\tableofcontents

\section{Problems}
Please track the time above to see whether this file is updated.

\subsection{6.3}
Using the Python function \texttt{sprime}, show that an average of $100$
random orientations for the stress/strain axis $l_i$ converges to the Reuss average.

\textit{Hint:} use the function
\texttt{2*np.pi*np.random.random((100,2))} to create random
values for spherical angles $\theta$, $\phi$ and generate $l_i$ from these angles.

\textbf{Solutions:}


\section{6.4}
Derive an expression for the p-wave modulus $M$ for different directions
of uniaxial strain defined by $l_i$.

\textbf{Solutions:}


\section{6.5}
Modify the Python function to calculate the p-wave modulus,
according to your result in the last problem.  Plot the modulus for a
full rotation of $2\pi$ in the \hkl(100) and \hkl(111) planes, and be sure to show that
your computed result agrees with analytic results for some low-index orientations.

\textbf{Solutions:}


\section{6.6}
Using the Landolt--Bornstein tables values, calculate the Voigt, Reuss,
and Voigt--Reuss--Hill average isotropic Young's moduls $E$ for \ce{Al}.

\textbf{Solutions:}


\section{6.7}
Derive the Young's modulus for the Reuss average of
a \hkl[111]-fiber-textured polycrystalline cubic solid.

\textbf{Solutions:}


% \bibliographystyle{unsrt}
% \bibliography{ref}

% --------------------------------------------------------------
%     You don't have to mess with anything below this line.
% --------------------------------------------------------------

\end{document}