% !TEX program = pdflatex
% !TEX encoding = UTF-8
% !TEX spellcheck = en_US
\documentclass[12pt]{article}

\usepackage[margin=1in]{geometry}
\usepackage[utf8]{inputenc}
\usepackage{amsmath,amsthm,amssymb}
\usepackage{graphicx}
\usepackage{hyperref} % Uso de links
\usepackage[version=4]{mhchem}
\usepackage{siunitx}
\usepackage{titlesec}
\usepackage{booktabs}
\usepackage{enumerate}
\usepackage{bm}
\usepackage{miller}
\usepackage{datetime}

\titleformat{\subsection}
  {\normalfont\large\bfseries}{}{0em}{}

\settimeformat{hhmmsstime}

\setlength{\parindent}{0em}

%\DeclareMathOperator{\tr}{tr}

\begin{document}

% --------------------------------------------------------------
%                         Start here
% --------------------------------------------------------------

\title{Mechanical Properties of Materials (MSAE 4215), Spring 2019\\ Homework 6 Solutions}
\author{Qi Zhang}
\date{\today, \currenttime}

\maketitle

\tableofcontents
\listoftables

\section{Problems}
Please track the time above to see whether this file is updated.

\subsection{7.3}
Determine the mode profile and resonant frequencies for a bar which is \textit{clamped}
at both ends, i.e. cannot be displaced in any direction, and fixed in both $x$ and $y$.

\textbf{Solutions:}

\subsection{7.4}
An earthquake strikes the San Francisco Bay Area;
how long does it take for the tremors to be felt in Los Angeles, \SI{600}{\kilo\meter} away?
Assume that the elastic waves propagate as through bulk (not surface) rock with density
of \SI{2.65}{\gram \per \cubic \centi \meter} and Young's modulus of \SI{10}{\giga\pascal}.
Calculate for the $P-$ (primary) wave due to compression and the
$S-$secondary wave due to shear, assuming a poisson ratio of $\nu= 0.25$.

\textbf{Solutions:}

\subsection{7.5}
Compare longitudinal wave velocities in a single crystal of \ce{Ag} for elastic wave
propagation along the \hkl[100] and \hkl[110] directions.  Is there a directional dependence of the velocity?
Why or why not?

\textbf{Solutions:}

\subsection{8.1}
Compare the effective Young's modulus for a $10\%$ volume fraction of hard material $A$ in a matrix of soft material $B$, $E_A=10 E_B$, in two geometries:
\begin{itemize}
  \item Fiber geometry (long axes of fibers along stress)
  \item Plate geometry (composition modulation in direction of primary stress), neglecting Poisson contraction
  \item Plate geometry (composition modulation in direction of primary stress), including Poisson contraction. Take $\nu_A= 0.3$, $\nu_B= 0.45$.
\end{itemize}

\textbf{Solutions:}
\begin{itemize}
  \item Using equation $(8.1)$,
        \begin{equation}
          E = v_A E_A + v_B E_B = 0.1 \times 10 E_B + 0.9 E_B = 1.9 E_B.
        \end{equation}

  \item Using equation $(8.2)$,
        \begin{equation}
          E = \frac{ 1 }{ \frac{ v_A }{ E_A } + \frac{ v_B }{ E_B } }
          = \frac{ 100 }{ 91 } E_B.
        \end{equation}

  \item Using equation $(8.17)$,
        \begin{equation}
          E^{-1} = \frac{V_A}{E_A}+\frac{V_B}{E_B}- \frac{2 V_A V_B}{E_A E_B}\frac{\left(\nu_A E_B-\nu_B E_A\right)^2}{V_A E_A\left(1- \nu_B\right)+ V_B E_B\left(1 - \nu_A\right)}
          \approx 0.64 / E_B,
        \end{equation}
        so $E = 1.5625 E_B$.
\end{itemize}

\subsection{8.2}
Derive a relationship for the Young's modulus of plates in the $x_1 - x_2$ plane,
$\sigma_1/\epsilon_1$ where the primary stress $\sigma_1$ is applied along one of
the in-plane directions of the plate. Express your answer in terms of the volume
fractions $V_A$, $V_B$ and isotropic elastic constants $E_A, E_B, \nu_A, \nu_B$.

\textbf{Solutions:}


\subsection{8.3}
Consider a model of a viscoelastic solid in which a Maxwell element $(E_a, \eta)$ is in
parallel (isostrain) with an elastic element $E_b$.
Derive the equation of motion relating $\epsilon,\sigma,\dot{\epsilon},\dot{\sigma}$.
Solve for the Young's modulus in the $t=0^+$ and $t=\infty$ limits, and plot
the strain $\varepsilon$ as a function of time under a step stress.

\textbf{Solutions:}
\begin{figure}[h]
  \centering
  \includegraphics[width=0.6\textwidth]{images/Maxwell}
  \caption{The Maxwell form of the standard linear solid.}
  \label{fig:Maxwell}
\end{figure}

The stress and strain in the Maxwell element will be denoted by $\sigma_M$ and $\varepsilon_M$.
The stress and strain in the elastic element will be referred to as $\sigma_b$ and $\varepsilon_b$.
Since the Maxwell element is in parallel with the elastic element (isostrain):
\begin{equation}
  \varepsilon_M = \varepsilon_b = \varepsilon,
\end{equation}
where $\varepsilon$ is the total strain of the SLS.
The strain in the Maxwell element is the sum of the strains in the spring and dashpot element, i.e.,
$\varepsilon_a$ and $\varepsilon_d$, so
\begin{equation}\label{eq:varepsilonall}
  \varepsilon_a + \varepsilon_d = \varepsilon_b = \varepsilon.
\end{equation}
In the same time,
\begin{equation}
  \dot{\varepsilon}_a + \dot{\varepsilon}_d = \dot{\varepsilon}_b = \dot{\varepsilon}.
\end{equation}
The total stress for the overall system is the sum of the stresses in the individual branches:
\begin{equation}\label{eq:sigmaall}
  \sigma = \sigma_b + \sigma_M = E_b \varepsilon_b + \sigma_M = E_b \varepsilon + \sigma_M,
\end{equation}
where in the Maxwell element,
\begin{equation}
  \sigma_M = \sigma_a = \sigma_d.
\end{equation}
We assume
\begin{align}\label{eq:sigmaM}
  \sigma_a & = E_a \varepsilon_a,          \\
  \sigma_d & = 3 \eta \dot{\varepsilon}_d.
\end{align}
And bring \eqref{eq:sigmaM} into \eqref{eq:varepsilonall},
we get
\begin{equation}
  \varepsilon = \frac{ \sigma_a }{ E_a } + \varepsilon_d = \frac{ \sigma_M }{ E_a } + \varepsilon_d.
\end{equation}
Take derivative on both sides,
\begin{equation}\label{eq:epdot}
  \dot{\varepsilon} = \frac{ \dot{\sigma}_M }{ E_a } + \frac{ \sigma_M }{ 3 \eta }.
\end{equation}
Take \eqref{eq:sigmaall} into \eqref{eq:epdot},
\begin{equation}\label{eq:beforesimplify}
  \dot{\varepsilon} = \frac{ \dot{\sigma} - E_b \dot{\varepsilon} }{ E_a } + \frac{ \sigma - E_b \varepsilon }{ 3 \eta }.
\end{equation}
Simplify equation \eqref{eq:beforesimplify}, we get the equation of motion
\begin{equation}\label{eq:maineq}
  \begin{split}
    \bigg( 1 + \frac{ E_b }{ E_a } \bigg) \dot{\varepsilon} + \frac{ E_b }{ 3 \eta } \varepsilon &= \frac{ \dot{\sigma} }{ E_a } + \frac{ \sigma }{ 3 \eta },\\
    (E_a + E_b) \dot{\varepsilon} + \frac{ E_a E_b }{ 3 \eta } \varepsilon &= \dot{\sigma} + \frac{ E_a }{ 3 \eta } \sigma,\\
    \dot{\varepsilon} + A \varepsilon &= B \dot{\sigma} + C \sigma,
  \end{split}
\end{equation}
where $A = \dfrac{ E_a E_b }{ E_a + E_b } \dfrac{ 1 }{ 3 \eta }$, $B = \dfrac{ 1 }{ E_a + E_b }$, and $C = A / E_b$.

Now let's solve this equation. You can skip this if not interested.
We apply Laplace transform
\begin{equation}
  \mathcal{F}(s) = \mathcal{L}\{f(t)\} = \int_{0}^{\infty} e^{-s t} f(t) d t
\end{equation}
to \eqref{eq:maineq}. Since Laplace transform is a linear operator, i.e.,
\begin{equation}
  \mathcal{L}\{a f(t)+b g(t)\} = a F(s) + b G(s),
\end{equation}
we will have
\begin{equation}\label{eq:lapexpand}
  \mathcal{L}\{\dot{\varepsilon}\} + A \mathcal{L}\{\varepsilon\} = B \mathcal{L}\{\dot{\sigma}\} + C \mathcal{L}\{\sigma\}.
\end{equation}
Since $\sigma(t)$ is a step function $\sigma(t) = \sigma_0 H(t)$, with $H(t)$ be the Heaviside step function,
\begin{equation}
  \mathcal{L}\{\sigma\} = \frac{ \sigma_0 }{ s }.
\end{equation}
And since
\begin{equation}\label{eq:lapdot}
  \mathcal{L}\{\dot{f}(t)\} = s F(s)-f(0^{+}),
\end{equation}
here if $f(t)$ is bounded by
\begin{equation}
  | f(t) | = M e^{a t},
\end{equation}
where $M$ is a finite number, then \eqref{eq:lapdot} simplifies to
\begin{equation}
  \mathcal{L}\{\dot{f}(t)\} = s F(s) - f(0).
\end{equation}
Of course in real world, $\varepsilon$ and $\sigma$ are finite, so
\begin{align}
  \mathcal{L}\{\dot{\sigma}\}      & = s \frac{ \sigma_0 }{ s } - \sigma(0) = 0,      \\
  \mathcal{L}\{\dot{\varepsilon}\} & = s \mathcal{L}\{\varepsilon\} - \varepsilon(0).
\end{align}
So \eqref{eq:lapexpand} simplifies to
\begin{equation}
  C \frac{ \sigma_0 }{ s } = s \mathcal{L}\{\varepsilon\} - \varepsilon(0) + A \mathcal{L}\{\varepsilon\},
\end{equation}
so
\begin{equation}\label{eq:transformed}
  \mathcal{L}\{\varepsilon\} = \frac{ C \frac{ \sigma_0 }{ s } + \varepsilon(0) }{ s + A }.
\end{equation}
Now what's $\varepsilon(0)$? Definitely it is a constant. Let's call it $\varepsilon_0$.

So apply inverse Laplace transform to \eqref{eq:transformed}, we have
\begin{equation}
  \varepsilon(t) = \frac{(E_b \varepsilon_0-\sigma_0) e^{-\frac{E_a E_b t}{3 \eta (E_a + E_b)}}}{E_b}+\frac{\sigma_0}{E_b},
\end{equation}
where $E_b \varepsilon_0 \leq \sigma_0$ since $\sigma_M(0) \geq 0$.
When $t \rightarrow \infty$, we can clearly see that
\begin{equation}
  \varepsilon(\infty) = \frac{\sigma_0}{E_b},
\end{equation}
as is expected.

So
\begin{equation}
  E(t) = \frac{ \varepsilon(t) }{ \sigma_0 } = \frac{(E_b \varepsilon_0-\sigma_0) e^{-\frac{E_a E_b t}{3 \eta (E_a + E_b)}}}{E_b \sigma_0}+\frac{1}{E_b},
\end{equation}
\begin{figure}[h]
  \centering
  \includegraphics[width=0.6\textwidth]{images/8_3}
  \caption{$\varepsilon(t)$, with different intial $\varepsilon_0$, where $E_a = 1$, $E_b = 3$, $\eta = 1$, $\sigma_0 = 10$.}
  \label{fig:question_8_3}
\end{figure}

\subsection{8.4}
The elongation of a cylinder of slightly viscoelastic material is found to be
$1\%$ greater for a step stress $\sigma_0$ measured after \SI{1}{\day} compared with after \SI{100}{\femto\second}.
If the viscoelasticity of the cylinder is dominated by a single defect
with a single relaxation time, what will be the \emph{shortest} time for a spring made from this
material set into free vibration to decay to $10\%$ of its initial amplitude,
if measurements are taken for a full range of masses applied to the end of the spring?

\textbf{Solutions:}

\subsection{8.5}
The notes on a piano span a range from \SI{27.5}{\hertz} to \SI{4186}{\hertz}.
The A keys are tuned to \SI{440}{\hertz} for A4, with each octave doubling in frequency
in the following sequence: 27.5, 55, 110, 220, 440, 880, 1760, 3520 Hz for A1-A8 on
the keyboard.  Plot (using python) as a function of frequency over $A0-A7$ \emph{the sustain time}
for a given note struck on the piano, defined as the time needed for the volume (square of amplitude)
to decay to $1/e$ its initial value.  Use a logarithmic scale for the frequency. Make your estimate
\begin{enumerate}[a)]
  \item assuming that the $Q$ value for the oscillation is that for $A4$ as calculated in the lectures,
  \item taking into account the frequency dependence of $Q$ expected for piano wire. Does the result seem reasonable?
\end{enumerate}

\textbf{Solutions:}

% \bibliographystyle{unsrt}
% \bibliography{ref}

% --------------------------------------------------------------
%     You don't have to mess with anything below this line.
% --------------------------------------------------------------

\end{document}