% !TEX program = pdflatex
% !TEX encoding = UTF-8
% !TEX spellcheck = en_US
\documentclass[12pt]{article}

\usepackage[margin=1in]{geometry}
\usepackage[utf8]{inputenc}
\usepackage{amsmath,amsthm,amssymb}
\usepackage{graphicx}
\usepackage{hyperref} % Uso de links
% \usepackage[version=4]{mhchem}
\usepackage{siunitx}
% \usepackage{enumerate}
\usepackage{titlesec}
% \usepackage{booktabs}
% \usepackage{bm}
\usepackage{miller}
\usepackage{datetime}

\titleformat{\subsection}
  {\normalfont\large\bfseries}{}{0em}{}
  
\settimeformat{hhmmsstime}

\setlength{\parindent}{0em}

%\DeclareMathOperator{\tr}{tr}

\begin{document}

% --------------------------------------------------------------
%                         Start here
% --------------------------------------------------------------

\title{Mechanical Properties of Materials (MSAE 4215), Spring 2019\\ Homework 4 Solutions}
\author{Qi Zhang}
\date{\today, \currenttime}

\maketitle

\tableofcontents

\section{Problems}
Please track the time above to see whether this file is updated.

\subsection{5.1}
Derive a relationship for the p-wave modulus $M$ in terms of the Young's modulus $E$ and Poisson ratio $\nu$.

\textbf{Solutions:}

We assume applying normal stresses $\sigma_{11}$, $\sigma_{22}$, and $\sigma_{33}$, which result in $\varepsilon_{11}$,
$\varepsilon_{22}$ and $\varepsilon_{33}$.
The stresses are chosen that their effects cancel each other, so
$2$ of the $\varepsilon$'s can be set to $0$ (by definition). Here we assume $\varepsilon_{22} = \varepsilon_{33} = 0$.

Write down above assumptions, the strains in $3$ directions are
\begin{align}
	\varepsilon_{11} & = \frac{\sigma_{11}}{E} - \nu \bigg( \frac{ \sigma_{22} }{ E } + \frac{ \sigma_{33} }{ E } \bigg), \label{eq:e11} \\
	0                & = \frac{\sigma_{22}}{E} - \nu \bigg( \frac{ \sigma_{11} }{ E } + \frac{ \sigma_{33} }{ E } \bigg), \label{eq:e22} \\
	0                & = \frac{\sigma_{33}}{E} - \nu \bigg( \frac{ \sigma_{11} }{ E } + \frac{ \sigma_{22} }{ E } \bigg). \label{eq:e33}
\end{align}
From $\eqref{eq:e22}$ we have
\begin{equation}\label{eq:s111}
	\sigma_{11} = \frac{ \sigma_{22} }{ \nu } - \sigma_{33}.
\end{equation}
From $\eqref{eq:e33}$ we have
\begin{equation}\label{eq:s112}
	\sigma_{11} = \frac{\sigma_{33}}{\nu} - \sigma_{22}.
\end{equation}
Combine $\eqref{eq:s111}$ and $\eqref{eq:s112}$ we have
\begin{equation}
	\sigma_{22} = \sigma_{33} = \frac{\nu}{1 - \nu} \sigma_{11}.
\end{equation}
Substitute above into $\eqref{eq:e11}$ we have
\begin{equation}
	\sigma_{11} = M \varepsilon_{11} = E \frac{1 - \nu}{1 - \nu - 2\nu^2} \varepsilon_{11}.
\end{equation}
So
\begin{equation}
	M = E \frac{1 - \nu}{1 - \nu - 2\nu^2}.
\end{equation}

\subsection{5.2}
Under what circumstances can the (volumetric) thermal expansion coefficient $\alpha$ be calculated from a single measurement of linear thermal expansion $\delta l/l$?

\textbf{Solutions:}

When the material is isotropic.
\begin{equation}
	\frac{ \delta l_i }{ l_i } \equiv \epsilon_1 = \epsilon_2 = \epsilon_3, \text{ where } i = 1, 2, 3.
\end{equation}
The dilation is
\begin{equation}
	\Delta = \frac{\Delta V}{V} = \epsilon_1 + \epsilon_2 + \epsilon_3 = 3 \epsilon.
\end{equation}
The volumetric thermal expansion coefficient is
\begin{equation}
	\alpha = \frac{1}{V} \bigg( \frac{\partial V}{\partial T} \bigg)_P = \bigg( \frac{ \Delta }{ \Delta T } \bigg)_P = 3 \bigg( \frac{ \epsilon }{ \Delta T } \bigg)_P.
\end{equation}

\subsection{5.3}
Gold has a Debye temperature of $\theta\sim\textrm{162 K}$, a Young's modulus of $E=\textrm{79 GPa}$, and a Poisson ratio of $\nu=0.43$.
Estimate its thermal expansion coefficient, making a reasonable assumption about $\gamma$.

\textbf{Solutions:}

According to \url{http://webelements.com/gold/crystal_structure.html}, the structure of gold crystal is FCC,
the cell parameter $a$ is \SI{0.4079}{\nano\meter}.

Because the Debye temperature of gold is low, at room temperature, gold is already under classical
limit. That is, the volumetric specific heat is
\begin{equation}
	C_V = 3 N k_B,
\end{equation}
where $N$ is the number of atoms in gold.

The thermal expansion coefficient is
\begin{equation}
	\alpha = \beta\gamma\frac{ C_V }{ V } = 3 k_B \beta\gamma\frac{ N }{ V } = 12 k_B \beta\gamma \frac{ 1 }{ a^3 },
\end{equation}
where $N = 4$, $V = a^3$ for an FCC cell.
We know from notes
\begin{equation}
	\beta = \frac{ 1 }{ K } = \frac{ 3 (1 - 2\nu) }{ E }.
\end{equation}
Here we just assume the Grüneisen parameter $\gamma$ of gold is $1$.
So
\begin{equation}
	\alpha = 36 k_B \frac{1 - 2\nu}{ E } \frac{ 1 }{ a^3 } \approx \SI{1.297e-5}{\per\kelvin},
\end{equation}
where $k_B = \SI{1.38e-23}{\joule\per\kelvin}$.

Well below \SI{162}{\kelvin}, under low-temperature limit,
\begin{equation}
	C_V =  N \frac{ 12 \pi^4 }{ 5 } k_B \bigg( \frac{ T }{ \theta } \bigg)^3,
\end{equation}
so
\begin{equation}
	\alpha = \frac{ 3 (1 - 2\nu) }{ E } \times N \frac{ 12 \pi^4 }{ 5 } k_B \bigg( \frac{ T }{ \theta } \bigg)^3 \frac{ 1 }{ a^3 }
	= T^3 \times \SI{2.38e-10}{\per\kelvin}.
\end{equation}

\subsection{5.4}
Derive an approximate expression for the Grüneisen parameter in terms of
anharmonicity $s$ and equilibrium interatomic spacing $d_0$ of the bonds, using the diatomic model.
({\it Hint:} how do the vibrational frequency and interatomic spacing change with temperature?)
You can assume the solid is isotropic and that the changes are small
(i.e. Taylor expand quantities where relevant.)  Use your results for Cu$_2$ from HW 1 to estimate
$\gamma$ in Cu. (The experimental value is $1.9$.)

\textbf{Solutions:}

For an isotropic material, using the result from question $5.2$,
\begin{equation}
	\alpha_V = \frac{ 1 }{ V } \Big( \frac{\partial V}{\partial T} \Big)_P = \frac{\partial \ln V}{\partial T} = 3 \alpha_i = -3 s \frac{ k_B }{ d_0 f }, \text{ where } i = 1, 2, 3.
\end{equation}

Using equation $(1.21)$ and $(1.22)$ from notes,
\begin{equation}
	\frac{ \omega^2 }{ \omega_0^2 } = 1 - s^2 \frac{ 2 k_B T }{ f },
\end{equation}
thus
\begin{equation}
	\frac{ \omega^2 }{ \omega_0^2 } = 1 + \frac{ 2 }{ 3 } \alpha d_0 s T.
\end{equation}
Therefore,
\begin{equation}
	\begin{split}
		\omega &= \omega_0 \sqrt{1 + \frac{ 2 }{ 3 } \alpha d_0 s T} \\
		&\approx \omega_0 \bigg( 1 + \frac{ 1 }{ 3 } \alpha d_0 s T \bigg).
	\end{split}
\end{equation}
since
\begin{equation}
	\sqrt{1 + x} \approx 1 + \frac{1}{2} x
\end{equation}
when $0 < x \ll 1$.
So
\begin{equation}
	\ln \omega = \ln \omega_0 + \ln \bigg( 1 + \frac{ 1 }{ 3 } \alpha d_0 s T \bigg) \approx \ln \omega_0 + \frac{ 1 }{ 3 } \alpha d_0 s T
\end{equation}
since
\begin{equation}
	\ln (1 + x) \approx x
\end{equation}
when $0 < x \ll 1$.
Thus,
\begin{equation}
	\Big( \frac{ \partial \ln \omega }{ \partial T } \Big)_P = \frac{ 1 }{ 3 } \alpha d_0 s.
\end{equation}
According to calculus,
\begin{equation}
	\Big( \frac{ \partial X }{ \partial Y } \Big)_{ Z } = \frac{ \Big( \frac{ \partial X }{ \partial W } \Big)_{ Z } }{ \Big( \frac{ \partial Y }{ \partial W } \Big)_{ Z } },
\end{equation}
thus
\begin{equation}
	\begin{split}
		\gamma &= -\frac{ \partial \ln \omega }{ \partial \ln V } \\
		&= -\frac{ \Big( \frac{ \partial \ln \omega }{ \partial T } \Big)_P }{ \Big( \frac{ \partial \ln V }{ \partial T } \Big)_P } \\
		&= -\frac{ \frac{ 1 }{ 3 } \alpha d_0 s }{ \alpha } \\
		&= -\frac{ d_0 s }{ 3 }.
	\end{split}
\end{equation}
From HW 1, $s = \SI{-2.040e10}{\per\meter}$, $d_0 = \SI{2.09e-10}{\meter}$, so
\begin{equation}
	\gamma = 1.4212.
\end{equation}


\subsection{6.1}
Determine an expression for the bulk modulus $K$ of a cubic crystal in terms of the matrix stiffnesses $c_{ij}$.
You can assume that the normal strain axes $\epsilon_i$ are along the unit cell axes.

\textbf{Solutions:}

$\beta$, the isothermal compressibility is defined as
\begin{equation}
	\beta = \frac{ 1 }{ K_T } = -\frac{ 1 }{ V }\Big( \frac{ \partial V }{ \partial P } \Big)_T
	= -\Big( \frac{ \Delta }{ \partial P } \Big)_T.
\end{equation}

The dilation, $\Delta$, is defined as,
\begin{equation}
	\Delta = \varepsilon_{11} + \varepsilon_{22} + \varepsilon_{33} = \epsilon_1 + \epsilon_2 + \epsilon_3.
\end{equation}
For a well-defined stress state, assuming that the normal strain axes $\epsilon_i$ are along the unit cell axes, we can express the strains
in terms of the compliances, for a cubic crystal, as
\begin{equation}
	\begin{bmatrix}
		\epsilon_1 \\
		\epsilon_2 \\
		\epsilon_3 \\
		\epsilon_4 \\
		\epsilon_5 \\
		\epsilon_6
	\end{bmatrix}
	=
	\begin{bmatrix}
		s_{11} & s_{12} & s_{12} & 0      & 0      & 0      \\
		s_{12} & s_{11} & s_{12} & 0      & 0      & 0      \\
		s_{12} & s_{12} & s_{11} & 0      & 0      & 0      \\
		0      & 0      & 0      & s_{44} & 0      & 0      \\
		0      & 0      & 0      & 0      & s_{44} & 0      \\
		0      & 0      & 0      & 0      & 0      & s_{44}
	\end{bmatrix}
	\begin{bmatrix}
		\sigma_1 \\
		\sigma_2 \\
		\sigma_3 \\
		\sigma_4 \\
		\sigma_5 \\
		\sigma_6
	\end{bmatrix}.
\end{equation}
Exerting a hydrostatic stress, $\sigma_1 = \sigma_2 = \sigma_3 = -P$,
\begin{equation}
	\begin{bmatrix}
		\epsilon_1 \\
		\epsilon_2 \\
		\epsilon_3 \\
		\epsilon_4 \\
		\epsilon_5 \\
		\epsilon_6
	\end{bmatrix}
	=
	\begin{bmatrix}
		s_{11} & s_{12} & s_{12} & 0      & 0      & 0      \\
		s_{12} & s_{11} & s_{12} & 0      & 0      & 0      \\
		s_{12} & s_{12} & s_{11} & 0      & 0      & 0      \\
		0      & 0      & 0      & s_{44} & 0      & 0      \\
		0      & 0      & 0      & 0      & s_{44} & 0      \\
		0      & 0      & 0      & 0      & 0      & s_{44}
	\end{bmatrix}
	\begin{bmatrix}
		-P \\
		-P \\
		-P \\
		0  \\
		0  \\
		0
	\end{bmatrix},
\end{equation}
then
\begin{equation}
	\epsilon_1 = \epsilon_2 = \epsilon_3 = -P (s_{11} + 2 s_{12}).
\end{equation}
So
\begin{equation}
	\Delta = -3P (s_{11} + 2 s_{12}).
\end{equation}
Therefore,
\begin{equation}
	\beta = -\frac{ \Delta }{ \partial P } = 3(s_{11} + 2 s_{12}),
\end{equation}
and
\begin{equation}
	K = \frac{ 1 }{ \beta } = \frac{ 1 }{ 3(s_{11} + 2 s_{12}) }.
\end{equation}

Using $(6.106)$ and $(6.107)$ from notes,
\begin{align}
	s_{11} & = \frac{c_{11}+c_{12}}{(c_{11} - c_{12})(c_{11}+2c_{12})}, \\
	s_{12} & = -\frac{c_{12}}{(c_{11} - c_{12})(c_{11}+2c_{12})},
\end{align}
we have
\begin{equation}
	K = \frac{ c_{11} + 2 c_{12} }{ 3 }.
\end{equation}

\subsection{6.2}
Using the Python function \texttt{sprime} distributed in class, plot the Young's modulus $s'_{1111}$ for Cu for a full rotation of stress and strain axes through
$2\pi$ in the \hkl(100) and \hkl(111) planes.
How do your plots obey Neumann's principle?

\textbf{Solutions:}

See the associated IPython notebook, or the PDF file. Thanks to Michael Carter's contribution.


% \bibliographystyle{unsrt}
% \bibliography{ref}

% --------------------------------------------------------------
%     You don't have to mess with anything below this line.
% --------------------------------------------------------------

\end{document}